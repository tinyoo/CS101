\documentclass[]{article}
\usepackage{lmodern}
\usepackage{amssymb,amsmath}
\usepackage{ifxetex,ifluatex}
\usepackage{fixltx2e} % provides \textsubscript
\ifnum 0\ifxetex 1\fi\ifluatex 1\fi=0 % if pdftex
  \usepackage[T1]{fontenc}
  \usepackage[utf8]{inputenc}
\else % if luatex or xelatex
  \ifxetex
    \usepackage{mathspec}
  \else
    \usepackage{fontspec}
  \fi
  \defaultfontfeatures{Ligatures=TeX,Scale=MatchLowercase}
\fi
% use upquote if available, for straight quotes in verbatim environments
\IfFileExists{upquote.sty}{\usepackage{upquote}}{}
% use microtype if available
\IfFileExists{microtype.sty}{%
\usepackage[]{microtype}
\UseMicrotypeSet[protrusion]{basicmath} % disable protrusion for tt fonts
}{}
\PassOptionsToPackage{hyphens}{url} % url is loaded by hyperref
\usepackage[unicode=true]{hyperref}
\hypersetup{
            pdfborder={0 0 0},
            breaklinks=true}
\urlstyle{same}  % don't use monospace font for urls
\usepackage{graphicx,grffile}
\makeatletter
\def\maxwidth{\ifdim\Gin@nat@width>\linewidth\linewidth\else\Gin@nat@width\fi}
\def\maxheight{\ifdim\Gin@nat@height>\textheight\textheight\else\Gin@nat@height\fi}
\makeatother
% Scale images if necessary, so that they will not overflow the page
% margins by default, and it is still possible to overwrite the defaults
% using explicit options in \includegraphics[width, height, ...]{}
\setkeys{Gin}{width=\maxwidth,height=\maxheight,keepaspectratio}
\IfFileExists{parskip.sty}{%
\usepackage{parskip}
}{% else
\setlength{\parindent}{0pt}
\setlength{\parskip}{6pt plus 2pt minus 1pt}
}
\setlength{\emergencystretch}{3em}  % prevent overfull lines
\providecommand{\tightlist}{%
  \setlength{\itemsep}{0pt}\setlength{\parskip}{0pt}}
\setcounter{secnumdepth}{0}
% Redefines (sub)paragraphs to behave more like sections
\ifx\paragraph\undefined\else
\let\oldparagraph\paragraph
\renewcommand{\paragraph}[1]{\oldparagraph{#1}\mbox{}}
\fi
\ifx\subparagraph\undefined\else
\let\oldsubparagraph\subparagraph
\renewcommand{\subparagraph}[1]{\oldsubparagraph{#1}\mbox{}}
\fi

% set default figure placement to htbp
\makeatletter
\def\fps@figure{htbp}
\makeatother


\date{}

\begin{document}

\section{Q1}\label{header-n0}

\begin{figure}
\centering
\includegraphics{C:/Users/lenovo/AppData/Roaming/Typora/typora-user-images/1575388420050.png}
\caption{}
\end{figure}

Main idea

\begin{itemize}
\item
  let \(i\) and \(j\) be the two edges of rectangles where \(i<j\), we
  have \(C=2(i+j),S=ij\)

  \[\frac{C^2}{S}=\frac{4(i^2+j^2+2ij)}{ij}=4(\frac{i}{j}+2+\frac{j}{i})\\
  \text{to minimize it, we need make $\frac{i}{j}$ close to 1,which also meani close to j as possible}\\\]

  for there are n pairs, we only need sort the n element by merge sort,
  and assign a temp variable \texttt{temp} to record the optimal ratio,
  then compare each two neighbor's ratio with \texttt{temp} if it is
  more close to 1, then update \texttt{temp}

  \begin{itemize}
  \item
    Note: we needing to pairing 2n edges first by sorting them and make 
  \end{itemize}
\end{itemize}

Algorithm

\begin{verbatim}
A={}

merge_sort(Array)

while i<2n-1 :

	if (Array[i]==Array[i+1]):

		A.append(Array[i])

		i=i+2

	else:

		i=i+1

	end if 

end while

merge_sort(A)

temp=A[0]/A[1]

for i from 1 to (A.size-1):

	if abs(A[i-1]/A[i]-1)<abs(temp-1):

    	temp=A[i-1]/A[i]

    end if 

end for
\end{verbatim}

correctness:

to make \(i/j\rightarrow1\), i and j must be close to each other in the
sorted array, and we have go through all the possibilities, so it can
find the optimal ratio which is closest to 1.

Analysis:

the merge sort is \(O(nlogn)\), and each comparation and update is
\(O(1)\), so total cost is \(n*O(1)+O(nlog(n))=O(nlog n)\)

\section{Q2}\label{header-n16}

\begin{figure}
\centering
\includegraphics{C:/Users/lenovo/AppData/Roaming/Typora/typora-user-images/1575393207332.png}
\caption{}
\end{figure}

main idea:

those children who are easy to satisfy is our priority target, we sort
two arrays Cake and Children in ascending order. and then initialize a
variable \texttt{num} to zero, then from children 1 to n, check if the
first element of Cake can satisfy such children, if not, remove the
first element of Cake. Keep comparing the first element of Cake and such
child, if satisfied, let's consider the next child when current child is
satisfied.

algorithm:

\begin{verbatim}
merge_sort(Cake)

merge_sort(Children)

ans=0

c=0

for i from 0 to n-1:

    while Children[i]>Cake[c]:

        c=c+1

        if c>=Cake.size:

            return ans

        end if 

    end while 

    ans=ans+1

end for 

return ans
\end{verbatim}

correctness: Let's say there are k children and m cakes. if all the m
cakes are given out, then such algorithm is optimal, if not, for
example, only m-1 children have satisfied, suppose our algorithm is not
optimal which means that there exists at least one children can be
satisfied. however the m-th child's expectation is more than the maximum
of cakes(that is why he is not satisfied), so it is a contradiction. so
the algorithm is correct. Analysis: Let n denote the number of cakes,
and m denote the number of children, the sorting of two arrays need
\(O(mlogm)+O(nlogn)\), and we also need to check all the n cakes
assigned to children once, which takes \(O(n)\), so the total cost is
\(O(nlog)+O(mlogm)\), which means if \(n\geq m\), then \(O(nlogn)\),
else \(O(m log m)\)

\section{Q3}\label{header-n23}

\begin{figure}
\centering
\includegraphics{C:/Users/lenovo/AppData/Roaming/Typora/typora-user-images/1575436464490.png}
\caption{}
\end{figure}

main idea:

check before the finish time of every program. record the last check
list with tags, if the current finished program is tagged, then we do
not need to check at that point.

The Arrays \texttt{Start} and \texttt{End} is the programs sorted in
ascending order. we need to check the early ending program, and the
programs whose start time ahead of it is checked.

algorithm:

\begin{verbatim}
P={programs}

Start=merge_sort(programs_start_time)

End=merge_sort(programs_end_time) 

count=0

check_points={}

checked=0



while count<n:

	if End[count].ifchecked==false:

		count=count+1

		check_points.append(End[count-1])

		checked=the index of the largest start time which is before End[count-1]

		//this step can be satrted from the Start[checked], which costs O(n) totally.

	end if 

return (count,check_points)

\end{verbatim}

correctness: Let's prove it by induction. if we only have 1 program, it
is easy to know the algorithm can get optimal value assume the algorithm
works when we have k algorithm, which is optimal. then if it overlapped
with previous programs, we do not need to check it, it is still optimal,
otherwise, it should be checked and we update count by 1, which is still
optimal. so proved.

Analysis: The sorting takes \(O(nlogn)\), we need to check n end points
in Start and add count to n, which costs \(O(n)\),so total cost is
\(O(nlogn) \)

\section{Q4}\label{header-n34}

\begin{figure}
\centering
\includegraphics{C:/Users/lenovo/AppData/Roaming/Typora/typora-user-images/1575437401157.png}
\caption{}
\end{figure}

main idea:

we sort the left and right requirements by merge sort into two arrays L
and R in ascending order.

Then set L{[}i{]}.guest to the right of R{[}i{]}.guest, if it is the
same one, just set them to an individual table.

algorithm

\begin{verbatim}
merge_sort(L)

merge_sort(R)

i=0

j=0

for i<n and j<n:

	if L[i].guest.assigned==true:

		i=i+1

	end if

	if R[j].guest.assigned==true:

		j=j+1

	end if

	if L[i].guest==R[j].guest

 		assign L[i].guest to an individual table

	else

 		assign L[i].guest to the right side of R[j].guest.

	end if 

end for 
\end{verbatim}

correctness:

Note that the question can be translated into such model, there are two
arrays, which can be combined into n pairs, the sum of chair is equal to
\(\sum\max\{pairs_i\}\), which \(pairs_i=(L_i,R_j)\)

first show you a example:

\[\left[\begin{matrix}

6\\

1\\

3\\

\end{matrix}\right]

\left[\begin{matrix}

2\\

5\\

8\\

\end{matrix}\right]\]

we know that 8 must be calculated. so we have such choice:

\[\left[\begin{matrix}

6\\

1\\

3\\

\end{matrix}\right]

\left[\begin{matrix}

8\\

\\

\\

\end{matrix}\right]

\\

\left[\begin{matrix}

6\\

1\\

3\\

\end{matrix}\right]

\left[\begin{matrix}

\\

8\\

\\

\end{matrix}\right]

\\

\left[\begin{matrix}

6\\

1\\

3\\

\end{matrix}\right]

\left[\begin{matrix}

\\

\\

8\\

\end{matrix}\right]\]

then the optimal choice must be make 6 and 8 a pair.

then we ignore \([6,8]\)

we have

\[\left[\begin{matrix}



1\\

3\\

\end{matrix}\right]

\left[\begin{matrix}

2\\

5\\



\end{matrix}\right]\]

which 5 must be calculated, the same way, make \(5,3\) into a pair is a
optimal choice

replace the concrete number with arbitrary letter i.e a,b,c\ldots{}..,
we can get a generalized version of prove.

by induction, we can observe that the optimal is sort L and R and make
the elements in same position into a pair, which means the two guys are
sitting with each other(include himself).

Analysis:

The sorting need \(2O(nlog(n))\), and the arrange need to go through
every guest which is \(O(n)\)

so the total complexity is \(O(nlogn) \)

\end{document}
