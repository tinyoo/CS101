\documentclass{article}
\usepackage[utf8]{inputenc}
\usepackage{natbib}
\usepackage{graphicx}
\usepackage{geometry}
\usepackage{color}
\usepackage{float}
\usepackage{hyperref}
\usepackage{enumerate}
\usepackage{fancyhdr}
\usepackage{titling}
\usepackage{amsmath}
\usepackage{amsmath,amssymb,amsthm}
\usepackage{listings}
\usepackage[table,xcdraw]{xcolor}
\usepackage{graphicx}
\renewcommand{\baselinestretch}{1.2}%Adjust Line Spacing
\newtheorem{Q}{Question}
\geometry{left=2.5cm,right=2.5cm,top=2.5cm,bottom=2.5cm}% Adjust Margins of the File
\usepackage{tikz-qtree}
\usetikzlibrary{graphs}
\tikzset{every tree node/.style={minimum width=2em,draw,circle},
	blank/.style={draw=none},
	edge from parent/.style=
	{draw,edge from parent path={(\tikzparentnode) -- (\tikzchildnode)}},
	level distance=1.2cm}
\setlength{\droptitle}{-6em}
%%% Code style
\lstset{
	%backgroundcolor=\color{red!50!green!50!blue!50},%代码块背景色为浅灰色
	rulesepcolor= \color{gray}, %代码块边框颜色
	breaklines=true,  %代码过长则换行
	numbers=left, %行号在左侧显示
	numberstyle= \small,%行号字体
	keywordstyle= \color{magenta},%关键字颜色
	commentstyle=\color{blue}, %注释颜色
	frame=shadowbox, %用方框框住代码块
	tabsize=3, %缩进大小
	showspaces = false
}
% Create horizontal rule command with an argument of height
\newcommand{\horrule}[1]{\rule{\linewidth}{#1}}
% Set the title here
\title{
    \normalfont \normalsize
    \textsc{ShanghaiTech University} \\ [25pt]
    \horrule{0.5pt} \\[0.4cm] % Thin top horizontal rule
    \huge CS101 Algorithms and Data Structures\\ % The assignment title
    \LARGE Fall 2019\\
    \LARGE Homework 5\\
    \horrule{2pt} \\[0.5cm] % Thick bottom horizontal rule
}
% wrong usage of \author, never mind
\author{}
\date{Due date: 23:59, October 27, 2019}

% set the header and footer
\pagestyle{fancy}
\lhead{CS101 Algorithms and Data Structures}
\chead{Homework 5}
\rhead{Due date: 23:59, October 27, 2019}
\cfoot{\thepage}
\renewcommand{\headrulewidth}{0.4pt}

% special settings for the first page
\fancypagestyle{firstpage}
{
	\renewcommand{\headrulewidth}{0pt}
	\fancyhf{}
	\fancyfoot[C]{\thepage}
}

% Add the support for auto numbering
% use \problem{title} or \problem[number]{title} to add a new problem
% also \subproblem is supported, just use it like \subsection
\newcounter{ProblemCounter}
\newcounter{oldvalue}
\newcommand{\problem}[2][-1]{
	\setcounter{oldvalue}{\value{secnumdepth}}
	\setcounter{secnumdepth}{0}
	\ifnum#1>0
		\setcounter{ProblemCounter}{#1}
	\else
		\stepcounter{ProblemCounter}
	\fi
	\section{Problem \arabic{ProblemCounter}: #2}
	\setcounter{secnumdepth}{\value{oldvalue}}
}
\newcommand{\subproblem}[1]{
	\setcounter{oldvalue}{\value{section}}
	\setcounter{section}{\value{ProblemCounter}}
	\subsection{#1}
	\setcounter{section}{\value{oldvalue}}
}

\begin{document}
\maketitle
\thispagestyle{firstpage}
%\newpage
\vspace{3ex}

\begin{enumerate}
\item Please write your solutions in English. 

\item Submit your solutions to gradescope.com.  

\item Set your FULL Name to your Chinese name and your STUDENT ID correctly in Account Settings. 

\item If you want to submit a handwritten version, scan it clearly. Camscanner is recommended. 

\item When submitting, match your solutions to the according problem numbers correctly. 

\item No late submission will be accepted.

\item Violations to any of above may result in zero score. 
\end{enumerate}
\newpage

\section{(5') Binary Search Tree and AVL Tree}
Each question has one or more correct answer(s). Select all the correct answer(s). For each question, you get $0$ point if you select one or more wrong answers, but you get $0.5$ point if you select a non-empty subset of the correct answers.\\
\textit{Note that you should write you answers of section 1 in the table below.}
\begin{table}[htbp]
	\begin{tabular}{|p{2cm}|p{2cm}|p{2cm}|p{2cm}|p{2cm}|}
		\hline 
		Question 1 & Question 2 & Question 3 & Question 4 & Question 5  \\ 
		\hline 
		&  &  &  & \\ 
		\hline 
	\end{tabular} 
\end{table}
\begin{Q}
	Consider the BST created by inserting the following keys in the given order in an initially empty BST: 1 5 7 2 3 6 4(ordered by value). Which other sequence(s), if any, produce the same BST?
	\begin{enumerate}[(A)]
		\item 3 1 2 5 7 4 6
		\item 1 6 5 7 4 2 3
		\item 1 5 2 7 3 4 6
		\item 1 5 6 7 2 4 3
	\end{enumerate}
\end{Q}


\begin{Q}
	Which of the followings are true? 
	\begin{enumerate}[(A)]
		\item For a min-heap inorder traversal gives the elements in ascending order.
		\item For a min-heap preorder traversal gives the elements in ascending order.
		\item For a BST inorder traversal gives the elements in ascending order.
		\item For a BST preorder traversal gives the elements in ascending order.
	\end{enumerate}
\end{Q}


\begin{Q}
	Which of the following statements are true for an AVL-tree?
	\begin{enumerate}[(A)]
		\item Inserting an item can unbalance non-consecutive nodes on the path from the root to the inserted item before the restructuring.
		\item Inserting an item can cause at most one node imbalanced before the restructuring.
		\item Removing an item in leaf nodes can cause at most one node imbalanced before the restructuring.
		\item Only at most one node-restructuring has to be performed after inserting an item.
	\end{enumerate}
\end{Q}



\begin{Q}
	Consider an AVL tree whose height is h, which of the following are true?
	\begin{enumerate}[(A)]
		\item this tree contains $\Omega(\alpha^h)$ keys, where $\alpha = \dfrac{1+\sqrt{5}}{2}$.
		\item this tree contains $\theta(2^h)$ keys.
		\item this tree contains $O(h)$ keys in the worst case.
		\item none of the above.
	\end{enumerate}
\end{Q}
\begin{Q}
Which of the following is TRUE?
\begin{enumerate}[(A)]
\item The cost of searching an AVL tree is $\theta(\log n)$ but that of a binary search tree is $O(n)$
\item The cost of searching an AVL tree is $\theta(\log n)$ but that of a complete binary tree is $\theta(n \log n)$
\item The cost of searching a binary search tree with height h is $O(h)$ but that of an AVL tree is $\theta(\log n)$
\item The cost of searching an AVL tree is $\theta(n log n)$ but that of a binary search tree is $O(n)$
\end{enumerate}
\end{Q}
%%%%% (2)
\section{(6') Magic BST}
Consider the binary search tree below. Each symbol represents an object stored in the BST. \\
\begin{minipage}{1\textwidth}
	\centering
	\begin{tikzpicture}
	\Tree
	[.$\alpha$
		[.$\beta$
			\edge[];[.$\gamma$
			]
			\edge[blank]; \node[blank]{};
		]
		[.$\lambda$
			\edge[];[.$\sigma$
			\edge[];\node[]{};
			\edge[blank]; \node[blank]{};
			]
			\edge[];[.$\omega$
			\edge[];[.$\theta$
			]
			\edge[];\node[]{};
			]
		]
	]
	\end{tikzpicture}
\end{minipage} \\
\begin{Q} (3') Based on the ordering given by the tree above, fill in the BST below with valid symbols. Symbols
must be unique. You may only use the 7 printed symbols (do not include any symbols from part b).
\begin{minipage}{1\textwidth}
	\centering
	\begin{tikzpicture}
	\Tree
	[.\node[]{};
		\edge[];[.\node[]{};
		\edge[];[.\node[]{};
		\edge[];[.\node[]{};]
		\edge[];[.\node[]{};
		\edge[blank];[.\node[blank]{};]
		\edge[];[.\node[]{};]
		]
		]
		\edge[];[.\node[]{};]
		]
		\edge[blank]; \node[blank]{};
	]
	\end{tikzpicture}
\end{minipage} \\
\end{Q}
\begin{Q} (3') For each of the insertion operations below, use the information given to "insert" the element into the
\textbf{TOP TREE WITH PRINTED SYMBOLS, NOT THE TREE WITH YOUR HANDWRITTEN 
SYMBOLS} by drawing the object (and any needed links) onto the tree. You can assume the objects are
inserted in the order shown below. You should not change anything about the original tree; you should
only add links and nodes for the new objects. If there is not enough information to determine where the
object should be inserted into the tree, circle “not enough information”. If there is enough information,
circle “drawn in the tree above” and \textbf{draw in the tree AT THE TOP OF THE PAGE.}

\begin{table}[htbp]
	\centering
	\begin{tabular}{cccc}
		insert($\phi$): & $\phi>\omega$ & Draw In Tree Above & Not enough Information \\ 
		insert($\chi$): & $\chi>\gamma$&  Draw In Tree Above & Not enough Information \\
		insert($\xi$): & $\alpha<\xi<\sigma$&  Draw In Tree Above & Not enough Information \\
		insert($\mu$): & $\lambda<\mu<\omega$&  Draw In Tree Above & Not enough Information \\
	\end{tabular} 
\end{table}
\end{Q}

\pagebreak
\section{(2')Property of BST}
\begin{Q}(2')
Teaching assistant Keyi thinks that he has discovered a critical property of the binary search tree. If you want to find a key in this tree, you will get a search path from the root to that key. Then he defines three sets of all the nodes in this tree. The first one $A$: the nodes are on the left to the searched node and not on the search path. The second one $B$: the nodes are on the search path. The third one $C$: the nodes are on the right to the searched node and not on the search path. Then he claims that any three keys: $a\in A$, $b\in B$ and $c\in C$ must satisfied $a\leq b\leq c$. Do you think his claim is right? If true, prove it. If not, give a counterexample. 
\end{Q}
\newpage

\section{(9')Only-child}
We define that the node is only-child if its parent node only have one children(Note: The root does not qualify as an only child). And we define a function of the binary tree: $OC(T)=$the number of only-child node.


\begin{Q}
(3')Prove that for any nonempty AVL tree T with n nodes, we have that OC(T)$\leq\dfrac{1}{2}n$
\end{Q}

\vspace{6cm}
\begin{Q}
(3')Is it true for any binary tree T with n nodes, that if $OC(T)\leq\dfrac{1}{2}n$ then $height(T)=O(\log n)$? If true, prove it. If not, give a counterexample.
\end{Q}
\vspace{6cm}
\begin{Q}
(3') Is it true for any binary tree T, that if there are $n$ node which is only-child, all of which are leaves, then $height(T)=O(\log n)$?
If true, prove it. If not, give a counterexample.
\end{Q}

\end{document}